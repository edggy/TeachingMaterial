\documentclass[11pt]{article}
\usepackage{amssymb}
\usepackage{amsmath}
\usepackage{enumerate}
\usepackage{hyperref}

\newcommand{\instructorname}{INSTRUCTOR NAME}
\newcommand{\instructoremail}{EMAIL@COLLEGE.edu}
\newcommand{\instructorphone}{(PHO) NEN-UMBE}
\newcommand{\instructoroffice}{OFFICE LOCATION}
\newcommand{\instructorhours}{Via Appointment}

\setlength{\textheight}{9in} \setlength{\headheight}{.2in}
\setlength{\headsep}{0in} \setlength{\topmargin}{0in}
\begin{document}
\begin{center}
MATH-110: Calculus I\\
Fall 2016\\
Siena College
\end{center}
%\vspace{.01in}

\begin{tabular}{r l}
\text{Instructor:} & \instructorname \\
\text{Email:} & \instructoremail \\
\text{Phone:} & \instructorphone\\
\text{Office:} & \instructoroffice\\
\text{Office Hours:} &\instructorhours \\
\text{Class:} &\text{Monday, Wednesday, Friday: 8:00 - 9:00 AM}\\
\text{Labs:} &\text{Tuesday, Thursday: 2:35 - 4:00 PM}\\
\text{Textbook:} &\text{Single Variable Calculus: Early Transcendentals Plus MyMathLab}\\
& By W. Briggs et al.\\
& Access Card Package, 2/E ISBN-10: 0321965175\\
\end{tabular}\\\\
%\vspace{.05in}

\vbox{\noindent {\bf Course Description} \par
	Courses MATH-110, MATH-120, and MATH-210 provide foundation for all upper level mathematics courses. Main topics considered during the first semester: functions, limits, continuity, differentiation, the chain-rule, anti-derivatives, the definite integral, Fundamental Theorem of Calculus and trigonometric functions. Applications of all topics are emphasized. Three hours of lecture and one hour and twenty-five minutes of laboratory per week. Lab fee. Students must purchase an approved graphing calculator prior to beginning this course. (ATTR: ARTS, CAQ)\\
}

\noindent {\bf Mission Statement and Learning Goals}
\begin{enumerate}
	\item \href{https://www.siena.edu/about/about-siena/mission}{Siena College (Mission Statement)}
	\item \href{https://www.siena.edu/academics/academics-at-siena/college-learning-goals}{Siena College (Learning Goals)}
	\item \href{https://www.siena.edu/academics/schools-departments/school-of-science/mission-statement-and-learning-goals/s}{School of Science (Mission Statement)}
	\item \href{https://community.siena.edu/academic-affairs/academics/school-of-science/departments/mathematics/mission-and-goals/}{The Mathematics Department (Mission and Goals)}
	\item \href{https://community.siena.edu/academic-affairs/resources/academic-integrity/academic-integrity-and-the-siena-student}{Academic Integrity}
\end{enumerate}
A student who has completed Calculus I and II successfully will:
\begin{enumerate}
	\item Understand the development of calculus as the solution to questions about rate and area and the role of calculus in science. (History)
	\item Be familiar with the concepts of limit, derivative and integral and their properties. Be able to perform computations involving these concepts. (Knowledge)
	\item Become familiar with the language of advanced mathematics. Be able to read, write and speak this language. (Communication)
	\item Reduce complicated problems in mathematics, the physical and life sciences and other disciplines to simple rules and procedures by applying the major concepts and theorems of calculus. (Modeling)
	\item Become familiar with the use of deductive reasoning in mathematics, be able to produce arguments involving several steps and to be able to explain each step in an easy to follow way. (Proofs)
	\item Be able to use a graphing calculator or computer to assist in understanding and solving calculus problems. (Technology)
\end{enumerate}

\vbox{\noindent {\bf Expectations} \par
	If you come to class I expect that you ask questions, participate and be respectful to myself and your classmates. You are welcome and encouraged to collaborate and engage each other while doing homeworks and labs. Be academically honest, i.e. if you use tools or resources you give credit to them. I encourage you to get familiar with these resources since they may come in handy in your everyday life.\\
}

\vbox{\noindent {\bf Attendance policy} \par
	Attendance is optional. It is highly recommended that you come to class, but if you have the skills to learn all the material elsewhere and still do well on the assessments then I have no problem with you doing so. On the other hand I may cover unique material or teach the material in a slightly different way that may create insight. Coming to class and being active in your learning, for example by asking questions, will make you a better student in the following years of your higher education.\\
}

\vbox{\noindent {\bf Office of Services for Students with Disabilities}\par
	From the Course Catalog: Siena College values the uniqueness of all students and is committed to supporting students with documented disabilities in order to provide them with reasonable accommodation that meets their individual needs. Students with disabilities are encouraged to register with the Office of Services for Students with Disabilities in Foy Hall upon acceptance to the college. Although students may choose to register at any time, accommodations are not retroactive. It is recommended that students register prior to July 15th for fall enrollment or December 15th for spring enrollment. To register with this office, a student must complete a student data sheet, release form, and provide current, comprehensive documentation of his/her disability. The office will determine the appropriateness of all documentation. Documentation must adequately represent the students current level of functioning as well as demonstrate the existence of a disability as required by the Americans with Disabilities Act and Section 504 of the Rehabilitation Act
	of 1973.
	
	All information disclosed during the registration process is considered private and will not be released without the students written permission. The college provides reasonable accommodations to all students with disabilities. Students are encouraged to meet with the Director of the Office of Services for Students with Disabilities to develop an individualized accommodation plan.
}

\vbox{\noindent {\bf Assessments and Grading Polices} \par
	There will be 3 exams, 2 quizzes, 13 labs, 9 homeworks and 1 final throughout the semester.
	
	Each exam will primarily assess your skills in the topics since the last exam, but may include questions from previous topics covered. Each exam will be worth 100 points
	
	Quizzes are there to ensure that you are keeping up with the material.  Quiz days will be broken up into 2 parts, the first part of class will be review, the second part will be the quiz. If you know your material the quizzes should take about 10 minutes to complete. Each quiz will be worth 50 points
	
	Each week there will be a lab assigned. The goal of the labs are to get you to think about the material in a different way and to collaborate with your peers. A lot of the questions will be subjective and will ask your opinion about the topics. Each lab will be graded from 0-11 points. 0 if it is not handed in, 1-10 based on how insightful your answers are, and 11 if you went above and beyond what was expected. (Note that the labs are out of 10 points so 11/10 would be a bonus)
	
	There will also be homework assigned each week. Each homework will be graded from 0-11 points. 0 if it is not handed in, 1-10 based on your answers, and 11 if you went above and beyond what was expected. (Each homework will be out of 10 points)
	
	The lowest homework and lab will be dropped.
	
	With all the assessments I expect you to clearly explain every step you take and cite any resources you used including your classmates.
	
	Your grade will be the sum of your exams, quiz, lab and homework grades.\\
	
	\begin{tabular}{r c l}
		\text{\bf Assesment} & \text{\bf Points Each} & \text{\bf Total Points}\\
		\text{Exams:} & 3 @ 100 & 300\\
		\text{Quizzes:} & 2 @ 50 & 100\\
		\text{Labs:} & 12/13 @ 10 & 120\\
		\text{Homeworks:} & 8/9 @ 10 & 80\\
		\text{Final:} & 1 @ 150 & 150\\
		\text{Total:} &  & 750\\
	\end{tabular}\\\\
}

\vbox{\noindent {\bf Academic Integrity} \par
	All violations of Academic Integrity will be taken very seriously. If I suspect that you are violating this policy, depending on the severity of the offense(s), I will either \begin{enumerate}[(a)] 
		\item Identify that you have violated the policy and give you a warning
		\item Deduct points from the offending assessment
		\item Fail you on the offending assessment
		\item Fail you for the entire class
	\end{enumerate}
	
	I believe that we are all adults here and hopefully I won't need to take any action regarding the violation of the Academic Integrity policy.\\
}

\vbox{\noindent {\bf Pandemic / Emergency Preparedness}
\begin{enumerate}[(a)]
	\item You are instructed to bring all texts and a copy of the syllabus home with you in the event of a College Closure. The Academic Calendar will be adjusted upon Reopening; so be prepared for the possibility of a short mini-semester; rescheduled class / exam period; and / or rescheduling of the semester, depending on the length of the closure.
	\item  If your situation permits, you should continue with readings and assignments to the best of your ability, per the course schedule (see the suggested schedule below).
	\item You will be given instructions regarding how to deal with the lab assignments requiring a graphing calculator (TI-89) by me, as needed.
	\item Online office hours will be used by me in order to maintain contact with you. You will be able to check-in with questions that you have. If you do not have Internet access available, I will also provide my home phone number, as needed. Remember, internet, mail delivery, and telephone services may also be impacted by a Pandemic or other emergency event.
	\item Finally, stay connected with information regarding the College's status and Reopening schedule by monitoring the Siena website, www.siena.edu.
\end{enumerate}

If a pandemic occurs that prevents you from being on campus, then you should go through the textbook and complete the online assignments based on the tentative schedule below. Please stay connected via e-mail and online office hours where I will be available to answer your questions and help you with the coursework.
}

\vbox{\noindent {\bf Tentative Class Schedule:} \\
\setlength{\tabcolsep}{12pt}
\begin{tabular}{c c c c}
\bf Date & \bf Topic & \bf Sections & \bf Homework Due\\
\hline
9/7 & Functions/Basic Sets & 1.1, 1.2 & \\
9/9 & Functions & 1.3, 1.4 & \\
\hline
9/12 & Limits & 2.1, 2.2 & HW1\\
9/14 & Limits & 2.3 & \\
9/16 & Limits & 2.4, 2.5& \\
\hline
9/19 & Limits & 2.6 & HW2\\
9/21 & Review & &\\
9/23 & Exam 1 & &\\
\hline
9/26 & Derivatives & 3.1 &\\
9/28 & Derivatives & 3.2 &\\
9/30 & Derivatives & 3.3 &\\
\hline
10/3 & Derivatives & 3.4 & HW3\\
10/5 & Derivatives & 3.5 &\\
10/7 & Derivatives & 3.6 &\\
\hline
10/10 & Derivatives & 3.7 & HW4\\
10/12 & Quiz 1 & & \\
10/14 & No Class & &\\
\hline
10/17 & Derivatives & 3.8 &\\
10/19 & Derivatives & 3.9 &\\
10/21 & Derivatives & 3.10 &\\
\hline
10/24 & Derivatives & 3.11 & HW5\\
10/26 & Review & &\\
10/28 & Exam 2 & &\\
\hline
10/31 & Applications of the Derivative & 4.1, 4.2 &\\
11/2 & Applications of the Derivative & 4.3 &\\
11/4 & Applications of the Derivative & 4.4 &\\
\hline
11/7 & Applications of the Derivative & 4.5 & HW6\\
11/9 & Quiz 2 & &\\
11/11 & Applications of the Derivative & 4.6 &\\
\hline
11/14 & Applications of the Derivative & 4.7 & HW7\\
11/16 & Applications of the Derivative & 4.9 &\\
11/18 & Review & &\\
\hline
11/21 & Exam 3 & &\\
11/23 & No Class & &\\
11/25 & No Class & &\\
\hline
11/28 & Integration & 5.1 &\\
11/30 & Integration & 5.2 &\\
12/2 & Integration & 5.3 &\\
\hline
12/5 & Integration & 5.4 & HW8\\
12/7 & Integration & 5.5 &\\
12/9 & Review & &\\
\hline
12/12 & Review & & HW9\\
\end{tabular}
}

\vbox{\noindent {\bf Tentative Lab Schedule:} \\
\setlength{\tabcolsep}{45pt}
\begin{tabular}{c c}
\bf Date & \bf Topic\\
\hline
9/6, 9/8 & Introduction\\
9/13, 9/15 & Fun Functions\\
9/20, 9/22 & Infinitesimals\\
9/27, 9/29 & Realistic Limits\\
\hline
10/4, 10/6 & Derivative is just the Slope?\\
10/11, 10/13 & Real Life Derivatives\\
10/18, 10/20 & Lots of Rules\\
10/25, 10/27 & Log and Trig Stuff\\
\hline
11/1, 11/3 & Extrema\\
11/8, 11/10 & Sketching\\
11/15, 11/17 & Mean Value\\
11/22 & No Lab\\
\hline
11/29, 12/1 & Riemann Sums\\
12/6, 12/8 & Fun Thm Calc\\
\end{tabular}
}


%\vbox{\noindent {\bf Problem 1 (a):} \\
%\begin{align*}
%\end{align*}
%}

\end{document}
